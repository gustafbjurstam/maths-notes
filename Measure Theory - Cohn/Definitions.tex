\documentclass[12pt]{article}

% Load AMS packages for advanced math
\usepackage{amsmath,amssymb,amsthm,mathrsfs}
\usepackage{pgfplots}
\usepackage{enumitem}
\pgfplotsset{compat=newest}

% Load geometry package and set margins
\usepackage[margin=1in]{geometry}

\newtheorem{lemma}{Lemma}
\newtheorem{sats}{Sats}

\theoremstyle{definition}
\newtheorem{definition}{Definition}[section]
\newtheorem*{remark}{Remark}

% Shortcuts
\newcommand{\N}{\mathbb{N}}      % natural numbers
\newcommand{\real}{\mathbb{R}}   % real numbers
\newcommand{\rat}{\mathbb{Q}}     % rationals
\newcommand{\eps}{\varepsilon}    % for nice epsilon
\newcommand{\sol}{ \noindent\textbf{Solution: } }   % creates Solution:
\newcommand{\prob}[1]{ \noindent\textbf{Problem #1.} }
\newcommand{\C}{\mathbb{C}}    % complex numbers
\renewcommand\part[1]{\vspace{.10in}\textbf{(#1)}}
\newcommand{\algebra}{\mathscr{A}}
\renewcommand{\L}{\mathscr{L}}
%\newtheorem*{solution}{Solution}
\newenvironment{solution}{\renewcommand{\proofname}{Lösning}\begin{proof}}{\end{proof}}
\newenvironment{bevis}{\renewcommand{\proofname}{Bevis}\begin{proof}}{\end{proof}}
\newcommand{\exempel}[1]{ \noindent\textbf{Exempel #1.} }
\usepackage{parskip}

\newcommand\norm[1]{\left\lVert#1\right\rVert}

%%%%%%%%%%%%%%%%%%%%%%%%%%%%%%%%%%%%%%%%%%%
\begin{document}

% Set linespacing to 1.5
\baselineskip 1.5em

\begin{center}
\textbf{\large Definitions used in Cohn's \textit{Measure Theory} (second edition)} \\
Gustaf Bjurstam\\
bjurstam@kth.se\\
\end{center}
\section{Measures}
\begin{definition}[Algebra]
    Let $X$ be a set, an arbitrary collection of subsets $\algebra$ of X is an \textit{algebra} on $X$ if
    \begin{enumerate}[label=(\alph*)]
        \item $X\in\algebra$,
        \item if $A\in\algebra$ then $A^c\in\algebra$,
        \item for each finite sequence $\{A_n\}_{n=1}^N$ of sets in $\algebra$, the set $\bigcup_{n=1}^N A_n$ belongs to $\algebra$, and
        \item for each finite sequence $\{A_n\}_{n=1}^N$ of sets in $\algebra$, the set $\bigcap_{n=1}^N A_n$ belongs to $\algebra$.
    \end{enumerate}
\end{definition}
\begin{definition}[$\sigma$-Algebra]
    Let $X$ be a set, an arbitrary collection of subsets $\algebra$ of X is a $\sigma$-\textit{algebra} on $X$ if
    \begin{enumerate}[label=(\alph*)]
        \item $X\in\algebra$,
        \item if $A\in\algebra$ then $A^c\in\algebra$,
        \item for each infinite sequence $\{A_n\}_{n=1}^\infty$ of sets in $\algebra$, the set $\bigcup_{n=1}^\infty A_n$ belongs to $\algebra$, and
        \item for each infinite sequence $\{A_n\}_{n=1}^\infty$ of sets in $\algebra$, the set $\bigcap_{n=1}^\infty A_n$ belongs to $\algebra$.
    \end{enumerate}
\end{definition}
\begin{definition}[Borel $\sigma$-algebra on $\real^d$]
    The \textit{Borel $\sigma$-algebra on $\real^d$}, denoted $\mathscr{B}(\real^d)$, is generated by the collection of open subsets of $\real^d$. \textbf{Proposition 1.1.5} states that $\mathscr{B}(\real^d)$ is generated by each of the collections of sets 
    \begin{enumerate}[label=(\alph*)]
        \item the collection of all closed subsets of $\real^d$;
        \item the collection of all closed half-spaces in $\real^d$ that have the form $\{(x_1,\dots,x_d):x_i\leq b\}$ for some $b\in\real$;
        \item the collection of all rectangles in $\real^d$ that have the form 
        \begin{equation*}
            \big\{(x_1,\dots,x_d):a_i<x_i\leq b_i\text{ for }i=1,\dots,d\big\}.
        \end{equation*}
    \end{enumerate}
\end{definition}
\begin{definition}[Measure]
    Let $\algebra$ be a $\sigma$-algebra. A function $\mu:\algebra\to[0,\infty]$ is called \textit{countably additive} if
    \begin{equation*}
        \mu\left(\bigcup_{n=1}^\infty A_n\right)=\sum_{n=1}^\infty \mu(A_n),
    \end{equation*}
    for each infinite sequence of disjoint sets $\{A_n\}_{n=1}^\infty$ in $\algebra$. If $\mu$ in addition to being countably additive also satisfies $\mu(\varnothing)=0$, $\mu$ is said to be a \textit{measure} on $\algebra$. 
\end{definition}
\begin{definition}[Measure space]
    Let $X$ be a set, $\algebra$ a $\sigma$-algebra on $X$ and $\mu$ a measure on $\algebra$. The triplet $(X,\algebra,\mu)$ is then called a \textit{measure space}, the pair $(X,\algebra)$ is often called a \textit{measurable space}.
\end{definition}
\begin{definition}[Outer measure]
    Let $X$ be a set, and let $\mathscr{P}(X)$ be the power set of $X$. An \textit{outer measure} on $X$ is a function $\mu^*:\mathscr{P}(X)\to [0,\infty]$ such that 
    \begin{enumerate}[label=(\alph*)]
        \item $\mu^*(\varnothing)=0$,
        \item $A\subseteq B\subseteq X$ implies $\mu^*(A)\leq \mu^*(B)$, and
        \item if $\{ A_n\}$ is an infinite sequence of sets in $\mathscr{P}(X)$, then $\mu^*\left(\bigcup A_n\right) \leq \sum \mu^*(A_n)$.
    \end{enumerate}
\end{definition}
\begin{definition}[Lebesgue outer measure]
    \textit{Lebesgue outer measure} on $\real^d$ which we denote by $\lambda^*$ is defined as follows. For each set $A\subseteq\real^d$ define the set $\mathscr{C}_A$ of all sequences $\{R_n\}$ of bounded and open $d$-cells $R_n$ such that $A\subseteq \bigcup_{n=1}^\infty R_n$. Then 
    \begin{equation*}
        \lambda^*(A)=\inf \left\{\sum_{n=1}^\infty \text{vol}(R_n):\{R_n\}\in \mathscr{C}_A\right\}.
    \end{equation*}
\end{definition}
\begin{definition}[$\mu^*$-measurable set]
    Let $X$ be a set, and let $\mu^*$ be an outer measure on $X$. A subset $B$ of $X$ is $\mu^*$-\textit{measurable} if 
    \begin{equation*}
        \mu^*(A)=\mu^*(A\cap B)+\mu^*(A\cap B^c),
    \end{equation*}
    for all $A\subseteq X$.
\end{definition}
\begin{definition}[Complete measure]
    Let $(X,\algebra,\mu)$ be a measure space. The measure $\mu$, or the measure space $(X,\algebra,\mu)$, is called \textit{complete} if $A\in\algebra$, $\mu(A)=0$, and $B\subseteq A$ implies $B\in\algebra$.
\end{definition}
\begin{definition}[$\mu$-negligible set]
    A subset $B$ of $X$ is called \textit{$\mu$-negligible} or \textit{$\mu$-null} if there exists $A\in\algebra$ such that $\mu(A)=0$ and $B\subseteq A$. Thus $(X,\algebra,\mu)$ is complete if and only if every $\mu$-negligible set belongs to $\algebra$.
\end{definition}
\begin{definition}[Completion of a $\sigma$-algebra under a measure]
    Let $(X,\algebra)$ be a measurable space. The \textit{completion} of $\algebra$ under $\mu$ is the collection $\algebra_\mu$ of $A\subseteq X$ for which there exists $E,F\in\algebra$ such that
    \begin{equation*}
        E\subseteq A\subseteq F,
    \end{equation*}
    and 
    \begin{equation*}
        \mu(F\setminus E)=0.
    \end{equation*}
    A set that belongs to $\algebra_\mu$ is sometimes said to be \textit{$\mu$-measurable}.
\end{definition}
\begin{definition}[Completion of a measure]
    Let $(X,\algebra,\mu)$ be a measure space. The \textit{completion} of $\mu$ is defined as $\overline{\mu}:\algebra_\mu\to [0,\infty]$ by letting $\overline{\mu}(A)$ be the common value of $E,F$, defined in the above definition.
\end{definition}
\begin{definition}[Inner and outer measure]
    Let $(X,\algebra,\mu)$ be a measure space, and let $A$ be an arbitrary subset of $X$. The \textit{inner measure} $\mu_*$ of $A$ is defined by
    \begin{equation*}
        \mu_*(A)=\sup\left\{\mu(B):B\subseteq A \text{ and } B\in\algebra\right\}.
    \end{equation*}
    The \textit{outer measure} $\mu^*$ of $A$ meanwhile, is defined by
    \begin{equation*}
        \mu^*(A)=\inf\left\{\mu(B):A\subseteq B \text{ and } B\in\algebra\right\}.
    \end{equation*}
\end{definition}
\begin{remark}
    According to \textbf{Proposition 1.5.4}, the outer measure defined in \textbf{Definition 1.13} satisfies the conditions placed on an outer measure in \textbf{Definition 1.6}. 
\end{remark}
\begin{definition}[Regular measure]
    Let $\algebra$ be a $\sigma$-algebra on $\real^d$ that includes $\mathscr{B}(\real^d)$. A measure $\mu$ on $\algebra$ is regular if 
    \begin{enumerate}[label=(\alph*)]
        \item each compact subset $K$ of $\real^d$ satisfies $\mu(K)<\infty$,
        \item each set $A$ in $\algebra$ satisfies
        \begin{equation*}
            \mu(A)=\inf\left\{\mu(U):U \text{ is open and} A\subseteq U\right\}, and
        \end{equation*}
        \item each open subset $U$ of $\real^d$ satisfies
        \begin{equation*}
            \mu(U)=\sup\left\{\mu(K):K \text{ is compact and} K\subseteq A\right\}.
        \end{equation*}
    \end{enumerate}
\end{definition}
\begin{definition}[Dykin class]
    Let $X$ be a set. A collection $\mathscr{D}$ is a \textit{d-system}, or \textit{Dykin class}, on $X$ if
    \begin{enumerate}[label=(\alph*)]
        \item $X\in \mathscr{D}$,
        \item $A\setminus B \in \mathscr{D}$ whenever $A,B\in\mathscr{D}$ and $A\supseteq B$, and
        \item $\bigcup A_n \in \mathscr{D}$ whenever $\{A_n\}$ is an increasing sequence of sets in $\mathscr{D}$.
    \end{enumerate}
\end{definition}
\begin{definition}[$\pi$-system]
    A collection of subsets of $X$ is a \textit{$\pi$-system} if it is closed under the formation of finite unions.
\end{definition}
\newpage
\section{Functions and Integrals}
\begin{definition}[$\algebra$-measurable function]
    Let $(X,\algebra)$ be a measurable space, and let $A\in \algebra$. A function $f:A\to[-\infty,\infty]$ is \textit{measurable with respect to $\algebra$} if it satisfies any of the conditions, and thus all, of the conditions in \textbf{Proposition 2.1.1}. That is any of
    \begin{enumerate}[label=(\alph*)]
        \item $\forall t\in \real \quad \{x\in A: f(x)\leq t\}\in\algebra$,
        \item $\forall t\in \real \quad \{x\in A: f(x)< t\}\in\algebra$,
        \item $\forall t\in \real \quad \{x\in A: f(x)\geq t\}\in\algebra$,
        \item $\forall t\in \real \quad \{x\in A: f(x)> t\}\in\algebra$.
    \end{enumerate}
    A function that is measurable with respect to $\algebra$ may be called \textit{$\algebra$-measurable} or if what $\sigma$-algebra is meant is obvious from context, simply \textit{measurable}. In the case $X=\real^d$ functions measurable with respect to $\mathscr{B}(\real^d)$ are called \textit{Borel measurable} or \textit{Borel functions}. A function measurable with respect to $\mathscr{M}_{\lambda^*}$ is called \textit{Lebesgue measurable}.
\end{definition}
\begin{definition}[Almost everywhere]
    Let $(X,\algebra,\mu)$ be a measure space. A property of points on $X$ is said to hold \textit{$\mu$-almost everywhere} if the set of points in $X$ where it fails to hold is $\mu$-negligible. The expression $\mu$-almost everywhere is often abbreviated $\mu$-a.e. or to a.e.$[\mu]$. If the measure is clear from context one may simply say \textit{almost everywhere}.
\end{definition}
\subsection{Construction of the integral}
\begin{definition}[Integral of a simple non-negative function]
    Let $\mu$ be a measure on $(X,\algebra)$. If $f$ is a real-valued, simple, $\algebra$-measurable function given by $f=\sum_{i=1}^m a_i \chi_{A_i}$, where each $a_i\geq0$ and $A_i\in\algebra$ are disjoint. Then the \textit{integral of $f$ with respect to $\mu$} is then defined to be
    \begin{equation*}
        \int f \,d\mu = \sum_{n=1}^m a_i \mu(A_i).
    \end{equation*}
\end{definition}
\begin{definition}[Integral of arbitrary $\algebra$-measurable, non-negative function]
    Let $f$ be an arbitrary $\algebra$-measurable function, with image in $[0,\infty]$. The integral of $f$ is then defined as
    \begin{equation*}
        \int f \,d\mu=\sup \left\{ \int g \,d\mu : g\in \mathscr{S}_+ \text{ and } g\leq f\right\}.
    \end{equation*}
\end{definition}
\begin{definition}[Integral of arbitrary measurable function]
    Let $f:X\to [-\infty,\infty]$ be a measurable function on $(X,\algebra,\mu)$. If $\int f^+ \,d\mu$ and $\int f^- \,d\mu$ are both finite, then $f$ is called \textit{integrable} and its \textit{integral} is defined by
    \begin{equation*}
        \int f \, d\mu=\int f^+ \, d\mu -\int f^- \, d\mu.
    \end{equation*}
    The integral of $f$ is said to \textit{exist} if at least one of $\int f^+ \,d\mu$ and $\int f^- \,d\mu$ is finite, in this case the integral is defined $\int f \, d\mu=\int f^+ \, d\mu -\int f^- \, d\mu$.
\end{definition}
\begin{definition}[Integral over a subset]
    Let $(X,\algebra,\mu)$ be a measure space, and let $f:X\to [-\infty,\infty]$ be $\algebra$-measurable. 
    The integral of $f$ over a subset $A\subseteq X$ is said to exist if the integral of $f\chi_A$ exists. In that case the integral over $A$ is defined to be
    \begin{equation*}
        \int_A f \, d\mu=\int f\chi_A \, d\mu.
    \end{equation*}
    Likewise, if $A\in \algebra$ and $f:A\to [-\infty,\infty]$ is $\algebra$-measurable, then the integral of $f$ over $A$ is defined to be the integral of the function which agrees with $f$ on $A$ and vanishes on $A^c$.
\end{definition}
\begin{definition}[Lebesgue integral]
    The case $X=\real^d$ and $\mu=\lambda$ we simply talk about \textit{Lebesgue integrability} and the \textit{Lebesgue integral}. We may use any of the following notations for the \textit{Lebesgue integral} over an interval $[a,b]$
    \begin{equation*}
        \int_a^b f=\int_a^b f(x) \, dx=(L)\int_a^b f=(L)\int_a^b f(x) \, dx,
    \end{equation*}
    where the latter two are used to emphasise that we are talking about the Lebesgue integral.
\end{definition}
\begin{definition}[$\L^1$]
    We define $\L^1(X,\algebra,\mu,\real)$, or sometimes simply $\L^1$, as the set of all integrable functions $f:X\to \real$. (As opposed to $[-\infty,\infty]$-valued functions.)
\end{definition}
\subsection{Measurable functions again}
\begin{definition}[Measurable function between sets]
    Let $(X,\algebra)$ and $(Y,\mathscr{B})$ be measurable spaces. A function $f:X\to Y$ is \textit{measurable with respect to $\algebra$ and $\mathscr{B}$} if for each $B\in \mathscr{B}$ the set $f^{-1}(B)$ belongs to $\algebra$. In stead of saying measurable with respect to $\algebra$ and $\mathscr{B}$, we may say that $f$ is a \textit{measurable function} from $(X,\algebra)$ to $(Y,\mathscr{B})$, or simply that $f:(X,\algebra)\to (Y,\mathscr{B})$ is \textit{measurable}.
\end{definition}
\begin{definition}[Integral of complex-valued function]
    Let $(X,\algebra,\mu)$ be a measure space. A complex-valued function $f$ on $X$ is \textit{integrable} if its real and imaginary parts $\Re (f)$ and $\Im (f)$ are integrable; if $f$ is integrable then its \textit{integral} is defined by
    \begin{equation*}
        \int f \, d\mu = \int \Re(f) \, d\mu +i\int \Im(f) \, d\mu.
    \end{equation*}
\end{definition}
\begin{definition}[$\mu f^{-1}$]
    Let $(X,\algebra,\mu)$ be a measure space, let $(Y,\mathscr{B})$ be a measurable space, and let $f:(X,\algebra)\to(Y,\mathscr{B})$ be measurable.Define $\mu f^{-1}:\mathscr{B}\to [0,\infty]$ by $\mu f^{-1}(B)=\mu(f^{-1}(B))$. It is easy to show that $\mu f^{-1}$ is a measure, this measure is sometimes called the \textit{image of $\mu$ under $f$}.
\end{definition}
\newpage
\section{Convergence}
\begin{definition}[Convergence in measure]
    Let $(X,\algebra,\mu)$ be a measure space, and let $f$ and $f_1,f_2,\dots$ be real valued $\algebra$-measurable functions on X. The sequence $\{f_n\}$ converges to $f$ \textit{in measure} if 
    \begin{equation*}
        \lim_n\mu\left(\left\{ x\in X: |f_n(x)-f_n|>\varepsilon\right\} \right)=0
    \end{equation*}
    for every $\eps>0$.
\end{definition}
\begin{definition}[Almost uniform convergence]
    Let $(X,\algebra,\mu)$ be a measure space, and let $f$ and $f_1,f_2,\dots$ be real valued $\algebra$-measurable functions on X. Then $\{f_n\}$ converges to $f$ \textit{almost uniformly} if for all $\eps>0$ there is $B\in \algebra$ such that $\{f_n\}$ converges to $f$ on $B$ and $\mu(B^c)<\eps$.
\end{definition}
\begin{definition}[Convergence in mean]
    Let $(X,\algebra,\mu)$ be a measure space, and let $f$ and $f_1,f_2,\dots$ be real valued $\algebra$-measurable functions on X. Then $\{f_n\}$ converges to $f$ \textit{in mean} if 
    \begin{equation*}
        \lim_n \int |f_n-f| \, d\mu =0.
    \end{equation*}
\end{definition}
\subsection{Normed spaces}
\begin{definition}[Norm \& seminorm]
    Let $V$ be a vector space over $\C$. A \textit{norm} on $V$ is a function $\norm{\cdot}:V \to \real$ that satisfies
    \begin{enumerate}[label=(\alph*)]
        \item $\norm{v}\geq 0$,
        \item $\norm{v}=0 \Longleftrightarrow v=0$,
        \item $\norm{\alpha v}=|\alpha|\norm{v}$,
        \item $\norm{u+v}\leq \norm{u}+\norm{v}$
    \end{enumerate}
    for each $u,v\in V$ and $\alpha\in \C$. If condition (b) was replaced by "$\norm{v}=0 \Longleftarrow v=0$" $\norm{\cdot}$ is a \textit{seminorm}.
\end{definition}
\begin{definition}[Metric \& semimetric]
    A \textit{metric} on a set $S$ is a function $d:S\times S\to \real$ that satisfies
    \begin{enumerate}[label=(\alph*)]
        \item $d(s,t)\geq 0$,
        \item $d(s,t)=0 \Longleftrightarrow s=t$,
        \item $d(s,t)=d(t,s)$,
        \item $d(r,t)\leq d(r,s)+d(s,t)$
    \end{enumerate}
    for all $r,s,t\in S$. If condition (b) is replaced by "$d(s,t)=0 \Longleftarrow s=t$" $d$ is a \textit{semimetric}. A \textit{metric space} is a set $S$ together with a metric $d$ on $S$. This may, if there is no risk for confusion with a measurable space, be written as $(S,d)$.
\end{definition}
\begin{definition}[Converging sequence]
    Let $(S,d)$ be a metric (or semimetric) space, a sequence $\{s_n\}$ in $S$ is said to \textit{converge} to $s\in S$ if for all $\eps>0$ there exists $N$ such that $\forall n \geq N$ $d(s_n,s)\leq\eps$. The point $s$ is then said to be the \textit{limit point} of $\{s_n\}$. In particular, if $V$ is a normed linear space, $v\in V$ and $\{v_n\}$ is a sequence in $V$, then $\{v_n\}$ converges to $v$ (with respect to the metric induced by the norm on $V$) if and only if $\lim_n \norm{v_n-v}=0$. Note that if $d$ is a semimetric $\{s_n\}$ may have several limit points.
\end{definition}
\begin{definition}[Dense subset]
    Let $(S,d)$ be a metric (or semimetric) space, a subset $A\subseteq S$ is said to be \textit{dense} in $S$ if for all $s\in S$ and $\eps > 0$ there exists $a\in A$ such that $d(s,a)<\eps$.
\end{definition}
\begin{definition}[Separable space]
    Let $(S,d)$ be a metric (or semimetric) space, if $S$ has a countable dense subset, $S$ is \textit{separable}.
\end{definition}
\begin{definition}[Cauchy sequences and completeness]
    Let $(S,d)$ be a metric space, a \textit{Cauchy sequence} is a sequence $\{s_n\}$ in $S$ such that for all $\eps>0$ there exists $N$ such that for all $n,m\geq N$, $d(s_n,s_m)<\eps$. A metric space $(S,d)$ is said to be \textit{complete} if all Cauchy sequences in $(S,d)$ converge.
\end{definition}
\begin{definition}[Banach space]
    If a normed linear space is complete, with respect to the metric induced by the norm on the space, then it is called a \textit{Banach space}.
\end{definition}
\begin{definition}[Inner product]
    Let $V$ be a vector space over $\C$. A function $(\cdot,\cdot):V\times V \to \C$ is an \textit{inner product} on $V$ if 
    \begin{enumerate}[label=(\alph*)]
        \item $(x,x)\geq0$,
        \item $(x,x)=0 \Longleftrightarrow x=0$,
        \item $(x,y)=\overline{(y,x)}$, and
        \item $(\alpha x+\beta y,z)=\alpha (x,z)+\beta (y,z)$
    \end{enumerate}
    hold for all $x,y,z\in V$ and $\alpha,\beta\in\C$. An \textit{inner product space} is a vector space, together with an inner product. The \textit{norm} $\norm{\cdot}$ associated to the inner product $(\cdot,\cdot)$ is defined by $\norm{x}=\sqrt{(x,x)}$. 
\end{definition}
\begin{definition}[Hilbert space]
    An inner product space that is complete under the norm associated with the inner product is called a \textit{Hilbert space}.
\end{definition}
\subsection{$\L^p$ and $L^p$}
\begin{definition}[$\mathscr{L}^p$]
    Let $(X,\algebra,\mu)$ be a measure space, and let $p\in [1,\infty)$. Then $\mathscr{L}^p(X,\algebra,\mu,\real)$ is the set of all $\algebra$-measurable functions $f:X\to \real$ such that $|f|^p$ is integrable, and $\mathscr{L}^p(X,\algebra,\mu,\C)$ is the set of $\algebra$-measurable functions $f:X\to\C$ such that $|f|^p$ is integrable.
\end{definition}
\begin{definition}[$\L^\infty$]
    Let $(X,\algebra,\mu)$ be a measure space. We define $\L^\infty(X,\algebra,\mu,\real)$ to be the set of all\footnote{I think it's supposed to be almost everywhere bounded functions, otherwise exercise 3.3.7 fails with this definition (however not with the alternative definition).} bounded real-valued $\algebra$-measurable functions, and $\L^p(X,\algebra,\mu,\C)$ as the set of all bounded complex-valued $\algebra$-measurable functions.
\end{definition}
\begin{remark}
    Some authors\footnote{Notably, the first edition of Cohn's \textit{Measure Theory} uses this definition.} define $\L^\infty(X,\algebra,\mu)$ as the set of all \textit{essentially bounded} $\algebra$-measurable functions on $X$. A function $f:X\to \C$ is \textit{essentially bounded} if there exists $M>0$ such that $\{x\in X: |f(x)|>M\}$ is locally $\mu$-null. For most purposes, it does not matter which definition of $\L^\infty$ one uses. However the study of liftings is convenient with \textbf{Definition 3.14}.
\end{remark}
\begin{definition}[Locally $\mu$-null]
    Let $(X,\algebra,\mu)$ be a measure space. A subset $N\subseteq X$ is said to be \textit{locally $\mu$-null} if for each $A\in\algebra$ that satisfies $\mu(A)<\infty$ the set $A\cap N$ is $\mu$-null. A property is said to hold \textit{locally almost everywhere} if the set on which the property doesn't hold is locally $\mu$-null.
\end{definition}
\begin{definition}[Seminorm on $\L^p$]
    In the case of $p\in [1,\infty)$ we define a seminorm $\norm{\cdot}_p:\L^p(X,\algebra,\mu)\to\real$ by
    \begin{equation*}
        \norm{f}_p=\left(\int |f|^p \, d\mu \right)^{1/p}.
    \end{equation*}
    In the case $p=\infty$ we define a seminorm $\norm{\cdot}_\infty:\L^\infty(X,\algebra,\mu)\to \real$ by
    \begin{equation*}
        \norm{f}_\infty = \inf \big\{ M: \{x\in X: |f(x)|>M\} \text{ is locally $\mu$-null}\big\}.
    \end{equation*}
\end{definition}
\begin{definition}[$\mathscr{N}^p$]
    Let $(X,\algebra,\mu)$ be a measure space, and let $\mathscr{N}^p(X,\algebra,\mu)$ be the subset of $\L^p(X,\algebra,\mu)$ which consists of the functions $f\in \L^p(X,\algebra,\mu)$ such that $\norm{f}_p=0$. That is, if $p\in[1,\infty)$, then $\mathscr{N}^p(X,\algebra,\mu)$ is the set of functions in $\L^p(X,\algebra,\mu)$ which vanish almost everywhere, and if $p=\infty$ then $\mathscr{N}^\infty(X,\algebra,\mu)$ is the set of functions in $\L^\infty(X,\algebra,\mu)$ which vanish locally almost everywhere.
\end{definition}
\newpage
\begin{definition}[$L^p$]
    Let $(X,\algebra,\mu)$ be a measure space. We define $L^p(X,\algebra,\mu)$ to be the quotient group $\L^p(X,\algebra,\mu)/\mathscr{N}^p(X,\algebra,\mu)$. That is $L^p(X,\algebra,\mu)$ is the collection of cosets of $\mathscr{N}^p(X,\algebra,\mu)$ in $\L^p(X,\algebra,\mu)$; these cosets are by definition the equivalence classes induced by the equivalence relation $\sim$, where $f\sim g$ holds if and only if $f-g\in\mathscr{N}^p(X,\algebra,\mu)$. Then if $p\in[1,\infty)$, $f\sim g\Longleftrightarrow f=g$ almost everywhere.
\end{definition}
\begin{definition}[Norm on $L^p$]
     Let $(X,\algebra,\mu)$ be a measure space. For each $f\in\L^p(X,\algebra,\mu)$ let $\langle f\rangle$ be the coset of $\mathscr{N}^p(X,\algebra,\mu)$ in $\L^p(X,\algebra,\mu)$ to which $f$ belongs. Then $L^p(X,\algebra,\mu)$ is a vector space and we can define a norm $\norm{\cdot}_p:L^p(X,\algebra,\mu)\to \real$ by $\norm{\langle f\rangle}_p=\norm{f}_p$, where on the right hand side $\norm{\cdot}_p:\L^p(X,\algebra,\mu)\to \real$ is given in \textbf{Definition 3.16}.
\end{definition}
\begin{definition}[Convergence in $p$th mean]
    Let $(X,\algebra,\mu)$ be a measure space, let $p\in [1,\infty)$, and let $f,f_1,f_2,\cdots \in \L^p(X,\algebra,\mu)$. Then $\{f_n\}$ converges to $f$ in $p$\textit{th mean}, or in $L^p$ \textit{norm}, if $\lim_n \norm{f_n-f}_p=0$.
\end{definition}
\subsection{Dual Spaces}
\begin{definition}[Linear operator]
    Let $V_1,V_2$ be normed vector spaces over $\C$ (or over $\real$), then a function $T:V_1\to V_2$ is a \textit{linear operator} or \textit{linear transformation} if $T(\alpha v)=\alpha T(v)$ and $T(u+v)=T(u)+T(v)$ hold for all $\alpha\in\C$ (or $\real$) and all $u,v\in V_1$.
\end{definition}
\begin{definition}[Bounded linear operator]
    Let $V_1,V_2$ be normed vector spaces, and let $T:V_1 \to V_2$ be linear. Then a nonnegative number $A$ such that $\norm{T(v)}\leq A\norm{v}$ holds for every $v\in V_1$ is called a \textit{bound} for $T$, and the operator $T$ is called \textit{bounded} if there is a bound for it.
\end{definition}
\begin{definition}[Norm of linear operator]
    Let $T:V_1\to V_2$ be a bounded linear operator, we define the \textit{norm} of $T$ by
    \begin{equation*}
        \norm{T}=\inf\{A: A\text{ is a bound for }T\}.
    \end{equation*}
    Then $\norm{\cdot}$ is a norm on the vector space of bounded linear operators from $V_1$ to $V_2$.
\end{definition}
\begin{definition}[Isometry]
    Let $T:V_1\to V_2$ be a linear operator between normed linear spaces. Then $T$ is called and \textit{isometry} if $\norm{T(v)}=\norm{v}$ for every $v\in V_1$.
\end{definition}
\begin{definition}[Isometric isomorphism]
    Let $T:V_1\to V_2$ be a linear operator between normed linear spaces. Then $T$ is an \textit{isometric isomorphism} if $T$ is an isometry and is surjective. Because all isometries are injective, $T$ is then bijective.
\end{definition}
\begin{definition}[Linear functional]
    Let $V$ be a normed linear space. A \textit{linear functional} on $V$ is a linear operator on $V$ whose values lie in $\C$, if $V$ is a vector space over $\C$, or in $\real$, if $V$ is a vector space over $\real$.
\end{definition}
\begin{definition}[Dual space]
    Let $V$ be a normed linear space. The set of all bounded, and hence continuous, linear functionals on $V$ then form a vector space. This vector space is called the \textit{dual space} (or \textit{conjugate space}) of $V$, and is denoted by $V^*$. Note that the function $\norm{\cdot}\:V^*\to \real$ which assigns to each functional in $V^*$ its norm, is in fact a norm on the vector space $V^*$.
\end{definition}
\newpage
\section{Signed and Complex measures}
\begin{definition}[Signed measure]
    Let $(X,\algebra)$ be a measurable space. A function $\mu:\algebra\to[-\infty,\infty]$ is called a \textit{signed measure} if it is countably additive and satisfies $\mu(\varnothing)=0$. 
\end{definition}
\begin{definition}[Positive \& negative sets]
    Let $\mu$ be a signed measure on a measurable space $(X,\algebra)$. A set $A\in \algebra$ is a \textit{positive set} if every $B\in\algebra$ such that $B\subseteq A$ satisfies $\mu(B)\geq 0$. Likewise, a set $A\in\algebra$ is a \textit{negative set} if every $B\in \algebra$ such that $B\subseteq A$ satisfies $\mu(B)\leq 0$.
\end{definition}
\begin{definition}[Hahn decomposition]
    A \textit{Hahn decomposition} of a signed measure $\mu$ on the measurable space $(X,\algebra)$ is a pair $(P,N)$ of disjoint subsets in $\algebra$ such that $X=P\cup N$, and $P$ is a positive set and $N$ is a negative set. Note that there may be several Hahn decomposition of the signed measure $\mu$.
\end{definition}
\begin{definition}[Complex measure]
    Let $(X,\algebra)$ be a measurable space. A \textit{complex measure} is a function $\mu:\algebra\to\C$ that satisfies $\mu(\varnothing)=0$ and is countably additive. A complex measure $\mu$ can be written as $\mu=\mu'+i\mu''$ where $\mu'$ and $\mu''$ are finite signed measures. 
\end{definition}
\begin{definition}[Jordan decomposition]
    Let $\mu$ be a signed measure on the measurable space $(X,\algebra)$, and let $(P,N)$ be a Hahn decomposition of $\mu$. Let $\mu^+(A)=\mu(A\cap P)$ and $\mu^-(A)=-\mu(A\cap N)$, then $\mu^+,\mu^-$ are measures on $(X,\algebra)$ and $\mu=\mu^+-\mu^-$. The measures $\mu^+$ and $\mu^-$ are called the \textit{positive part} and \textit{negative part} of $\mu$, respectively. The representation $\mu=\mu^+-\mu^-$ is called the \textit{Jordan decomposition} of the signed measure $\mu$. If $\mu$ is a complex measure on $(X,\algebra)$ then the representation $\mu=\mu_1-\mu_2 +i(\mu_3-\mu_4)$ is called the \textit{Jordan decomposition} of $\mu$, if $\mu'=\mu_1-\mu_2$ and $\mu''=\mu_3-\mu_4$ are the Jordan decompositions of the real and imaginary parts of $\mu$.
\end{definition}
\begin{definition}[Variation]
    If $\mu$ is a signed measure on the measurable space $(X,\algebra)$, then the \textit{variation} of $\mu$ is defined to be $|\mu|=\mu^++\mu^-$, and the \textit{total variation} of $\mu$ is defined to be $\norm{\mu}=|\mu|(X)$. If $\mu$ is a complex measure on $(X,\algebra)$, then the \textit{variation} of $\mu$ is defined by 
    \begin{equation*}
        |\mu|(A)=\sup\left\{\sum_{j=1}^n|\mu(A_j)|: \{A_j\}_{j=1}^n \text{ are finite disjoint sequences in $\algebra$ such that } A=\bigcup_{j=1}^n A_j\right\}.
    \end{equation*}
    The total variation of $\mu$ is defined to be $\norm{\mu}=|\mu|(X)$.
\end{definition}
\begin{definition}
    Let $(X,\algebra)$ be a measurable space. Define $M(X,\algebra,\real)$ as the set of all finite signed measures on $(X,\algebra)$, and $M(X,\algebra,\C)$ as the set of all complex measures on $(X,\algebra)$. It is easy to see that $M(X,\algebra,\real)$ and $M(X,\algebra,\C)$ are vector spaces over $\real$ and $\C$ respectively, and that the total variation gives a norm on each of them.
\end{definition}
\begin{definition}[Integration with signed measure]
    Let $X,\algebra)$ be a measurable space. Denote by $B(X,\algebra,\real)$ the vector space of bounded real-valued $\algebra$-measurable functions on $X$. If $\mu$ is a finite signed measure on $(X,\algebra)$, and $\mu=\mu^+-\mu^-$ is the Jordan decomposition of $\mu$, and if $f\in B(X,\algebra,\real)$, then the \textit{integral of $f$ with respect to $\mu$} is defined as
    \begin{equation*}
        \int f \, d\mu=\int f\, d\mu^+-\int f\, d\mu^-.
    \end{equation*}
\end{definition}
\begin{definition}[Integration with complex measure]
    Let $X,\algebra)$ be a measurable space. Denote by $B(X,\algebra,\C)$ the vector space of bounded complex-valued $\algebra$-measurable functions on $X$. If $\mu$ is a complex measure on $(X,\algebra)$, and $\mu_1,\mu_2$ are the real and imaginary parts of $\mu$, and if $f\in B(X,\algebra,\C)$, then the \textit{integral of $f$ wuth respect ti $\mu$} is defined by
    \begin{equation*}
        \int f\,d\mu=\int f\,d\mu_1+i\int f \,d\mu_2.
    \end{equation*}
\end{definition}
\begin{remark}
    The formula $f\mapsto \int f\,d\mu$ and $\mu\mapsto \int f\,d\mu$ define a linear functionals on $B(X,\algebra)$ and $M(X,\algebra)$ respectively.
\end{remark}
\begin{definition}[Absolute continuity]
    Let $(X,\algebra)$ be a measureable space, and let $\mu$ and $\nu$ be measures on $(X,\algebra)$. We say that $\nu$ \textit{is absolutely continuous with respect to} $\mu$ if every $A\in \algebra$ such that $\mu(A)=0$ also satisfies $\nu(A)=0$. This is sometimes indicated as $\nu\ll\mu$. A measure $\nu$ on $(\real^d,\mathscr{B}(\real^d))$ is called absolutely continuous if $\nu\ll\lambda$.
\end{definition}
\begin{definition}[Absolute continuity of signed or complex measure]
    Let $(X,\algebra,\mu)$ be a measure space. A signed or complex measure $\nu$ on $(X,\algebra)$ is \textit{absolutely continuous with respect to} $\mu$, written $\nu\ll\mu$, if the variation $|\nu|$ is absolutely continuous with respect to $\mu$.
\end{definition}
\begin{definition}[Radon-Nikodym derivative]
    Let $(X,\algebra)$ be a measurable space, let $\mu$ be a $\sigma$-finite meaure on $(X,\algebra)$ and let $\nu$ be a, finite signed, complex, or $\sigma$-finite, meaure on $(X,\algebra)$ such that $\nu\ll\mu$. A function $g$ such that $\nu(A)=\int_Ag\,d\mu$ hold for every $A\in\algebra$ is called a \textit{Radon-Nikodym derivative} of $\nu$ with respect to $\mu$, or in light of the $\mu$-almost uniqueness of such $g$, \textit{the Radon-Nikodym derivative} of $\nu$ with respect to $\mu$. The Radon-Nikodym derivative of $\nu$ is sometimes denoted $\frac{d\nu}{d\mu}$.
\end{definition}
\begin{definition}[Concentrated measure]
    Let $(X,\algebra)$ be a measurable space, a measure $\mu$ is \textit{concentrated on} $E\in\algebra$ if $\mu(E^c)=0$. A signed or complex measure $\mu$ is said to be concentrated on $E$ if $|\mu|(E^c)=0$.
\end{definition}
\begin{definition}[Singularity]
    Let $(X,\algebra)$ be a measurable space, let $\mu$ and $\nu$ be positive, signed, or complex measures on $(X,\algebra)$. Then $\mu$ and $\nu$ are called \textit{mutually singular} if there exists $E\in\algebra$ such that $\mu$ is concentrated on $E$ and $\nu$ is concentrated on $E^c$. That two measures are mutually singular is sometimes denoted $\mu\perp\nu$. Sometimes the statement $\mu$ and $\nu$ are mutually singular is said, $\mu$ and $\nu$ are singular, $\mu$ is singular with respect to $\nu$, or that $\nu$ is singular with respect to $\mu$. A positive, signed, or complex measure on $(\real^d,\mathscr{B}(\real^d))$ is simply said to be \textit{singular} if it is singular with respect to the $d$-dimensional Lebesgue measure on $(\real^d,\mathscr{B}(\real^d))$.
\end{definition}
\begin{definition}[Lebesgue decomposition]
    Let $(X,\algebra,\mu)$ be a measure space, and let $\nu$ be a finite signed, complex, or $\sigma$-finite positive measure on $(X,\algebra)$. There are unique finite signed, complex, or $\sigma$-finite measures $\nu_a$ and $\nu_s$ on $(X,\algebra)$ that satisfy
    \begin{enumerate}[label=(\alph*)]
        \item $\nu_a\ll\mu$,
        \item $\nu_s\perp\mu$, and
        \item $\nu=\nu_a+\nu_s$.
    \end{enumerate}
    The decomposition $\nu=\nu_a+\nu_s$ is called \textit{the Lebesgue decomposition of $\nu$}.
\end{definition}
\newpage
\section{Product Measures}
\begin{definition}[Product of $\sigma$-algebras]
    Let $(X,\algebra)$ and $(Y,\mathscr{B})$ be measurable spaces. A subset of $X\times Y$ is called a \textit{rectangle with measurable sides} if it has the form $A\times B$ for some $A\in\algebra$ and $B\in\mathscr{B}$. The $\sigma$-algebra on $X\times Y$ generated by collection of rectangles with measurable sides is called the \textit{product} of $\algebra$ and $\mathscr{B}$, and is denoted by $\algebra\times \mathscr{B}$.
\end{definition}
\begin{definition}[Product measure]
    Let $(X,\algebra,\mu)$ and $(Y,\mathscr{B},\nu)$ be $\sigma$-finite measure spaces. The unique measure $\mu\times\nu$ on $\algebra\times\mathscr{B}$ which satisfies $(\mu\times\nu)(A\times B)=\mu(A)\nu(B)$, for every $A\in\algebra,B\in\mathscr{B}$, is called the \textit{product} of $\mu$ and $\nu$.
\end{definition}
\newpage
%%BELOW IS SOME PROBABILITY
\setcounter{section}{9}
\section{Probability}
\begin{definition}[Probability space]
    A \textit{probability space} is a measure space $(\Omega,\algebra,P)$ such that $P(\Omega)=1$. The elements of $\Omega$ are called the \textit{elementary outcomes} or the \textit{sample points} of our experiment, and the members of $\algebra$ are called \textit{events}. If $A\in\algebra$, then $P(A)$ is the \textit{probability} of the event $A$.
\end{definition}
\begin{definition}[Random variable]
    A \textit{real-valued random variable} on a probability space $(\Omega,\algebra,P)$ is an $\algebra$-measurable function from $\Omega$ to $\real$. Such a variable represents a numerical observation or measurement whose value depends on the outcome of the random experiment represented by $(\Omega,\algebra,P)$. More generally, a \textit{random variable} with values in a measurable space $(S,\mathscr{B})$ is a measurable function from $(\Omega,\algebra,P)$ to $(S,\mathscr{B})$.
\end{definition}
\begin{definition}[Distribution]
    Let $X$ be a random variable with values in $(S,\mathscr{B})$. The the \textit{distribution} of $X$ is the measure $PX^{-1}$ (see \textbf{Definition 2.11}) defined on $(S,\mathscr{B})$ by  $(PX^{-1})(A)=P(X^{-1}(A))$. We will often write $P_X$ for the distribution of a random variable $X$. If $X_1,\dots,X_d$ are $(S,\mathscr{B})$-valued random variables on $(\Omega,\algebra,P)$, then the formula $X(\omega)=(X_1(\omega,\dots,X_d(\omega))$ defines an $S^d$-valued random variable $X$; the distribution of $X$ is called the \textit{joint distribution} of $X_1,\dots,X_d$.
\end{definition}
\begin{definition}[Expected value]
    If a real-valued random variable on the probability space $(\Omega,\algebra,P)$ is integrable, then the \textit{expected value} of $X$ is defined $E(X)=\int X\,dP$. The expected value of $X$ is often denoted $\mu_X$.
\end{definition}
\begin{definition}[Variance]
    If $X$ is a real-valued random variable, then the \textit{variance} of $X$ is the expected value of the random variable $(X-E(X))^2$, often denoted $\text{Var}(X)$ or $\sigma_X^2$. The numerical value $\sqrt{\sigma_X^2}=\sigma_X$ is called the \textit{standard deviation} of $X$.
\end{definition}
\begin{definition}
    If $X$ is $\real^d$ valued and $P_X\ll\lambda$, then the Radon-Nikodym derivative of $P_X$ $f_X$, is called the \textit{density function} of $X$.
\end{definition}
\begin{definition}[Independence]
    Let $(\Omega,\algebra,P)$ be a probability space, and led $\{A_i\}_{i\in I}$ be an indexed family of events in $\algebra$. The events $A_i$ are called \textit{independent} if for each finite subset $I_0$ of $I$ we have $P(\cap_{i\in I_0} A_i)=\prod_{i\in I_0} P(A_i)$. Let $\{X_i\}_{i\in I}$ be an indexed family of random variables defined on $(\Omega,\algebra,P)$ and with values in the measurable space $(S,\mathscr{B})$. The random variables $X_i$ are aclled \textit{independent} if for each choice of sets $A_i\in\mathscr{B}$, $i\in I$, the events $X_i^{-1}(A_i)$ are independent. Finally if $\{\algebra\}_{i\in I}$ is an indexed family of sub-$\sigma$-algebras of $\algebra$, then the $\sigma$-algebras $\algebra_i$ are independent if for each choice $A_i\in\algebra_i$ the events $A_i$ are independent.
\end{definition}
\end{document}
