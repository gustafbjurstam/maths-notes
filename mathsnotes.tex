% Basic preamble with commands and packages

% Document related packages
\usepackage[margin=1in, a4paper]{geometry}
\usepackage{parskip}
\usepackage{enumitem}
\usepackage{xcolor}
\baselineskip 1.5em

% AMS packages and other math fonts
\usepackage{amsmath}
\usepackage{amssymb}
\usepackage{amsthm}
\usepackage{mathrsfs}
\usepackage{amsfonts}
\usepackage{bbm}

% Theorem and proof environments
% By default, all theorems, propositions, lemmas, etc. should share a counter
\theoremstyle{plain}
\newtheorem{theorem}{Theorem}[section]
\newtheorem{lemma}[theorem]{Lemma}
\newtheorem{proposition}[theorem]{Proposition}
\newtheorem{corollary}[theorem]{Corollary}

\theoremstyle{definition}
\newtheorem{definition}{Definition}[section]
\newtheorem*{remark}{Remark}

% Shortcuts
% Specific sets
\newcommand{\N}{\mathbb{N}}
\newcommand{\real}{\mathbb{R}}
\newcommand{\rat}{\mathbb{Q}}
\newcommand{\eps}{\varepsilon}
\newcommand{\C}{\mathbb{C}}
\newcommand{\D}{\mathbb{D}}
\renewcommand{\H}{\mathbb{H}}
\newcommand{\Z}{\mathbb{Z}}
\renewcommand{\empty}{\varnothing}

% Various mathscr/mathcal etc, algebras etc.
\newcommand{\F}{\mathscr{F}}    % Fourier
\newcommand{\algebra}{\mathscr{A}}
\newcommand{\Lscr}{\mathscr{L}}
\newcommand{\Lcal}{\mathcal{L}}
\newcommand{\Acal}{\mathcal{A}}
\newcommand{\Ecal}{\mathcal{E}}


% Parentases, sets, norms, etc.
\newcommand{\norm}[1]{\left\lVert#1\right\rVert}
\newcommand{\inner}[2]{\left\langle#1, #2\right\rangle}
\newcommand{\abs}[1]{\left|#1\right|}
\newcommand{\set}[1]{\left\{#1\right\}}
\newcommand{\Fo}[1]{\F\left[#1\right]} % Fourier transform
\newcommand{\Foi}[1]{\F^{-1}\left[#1\right]} % Inverse fourier
\newcommand{\E}[1]{\mathbb{E} \left[#1\right]} % Expected value

% Other notations
\newcommand{\closure}[1]{\overline{#1}}
\newcommand{\dual}{{}^{\star}}
\newcommand{\bidual}{{}^{\star\star}}

% Operators
\DeclareMathOperator{\cond}{cond}
\DeclareMathOperator{\range}{range}
\DeclareMathOperator{\domain}{domain}
\DeclareMathOperator{\argmin}{argmin}
\DeclareMathOperator{\prox}{prox}
\DeclareMathOperator{\TV}{TV}
\DeclareMathOperator{\TGV}{TGV^2_{\alpha,\beta}}
\DeclareMathOperator{\BGV}{BGV^2_{\alpha,\beta}}
\let\div\relax
\DeclareMathOperator{\div}{div}
\DeclareMathOperator{\var}{Var}
\DeclareMathOperator{\cov}{Cov}
\DeclareMathOperator{\rank}{rank}
\DeclareMathOperator{\epi}{epi}

% Make it possible to redefine things
% theorems
\makeatletter
\def\cleartheorem#1{%
  % Undefine the environment start command (e.g., \lemma)
  \expandafter\let\csname#1\endcsname\relax
  % Undefine the environment end command (e.g., \endlemma)
  \expandafter\let\csname end#1\endcsname\relax
  % Undefine the counter (e.g., \c@lemma)
  \expandafter\let\csname c@#1\endcsname\relax
  % Undefine the numbering macro (e.g., \thelemma) - THIS IS THE MISSING PART
  \expandafter\let\csname the#1\endcsname\relax
}
\makeatother
\def\clearthms#1{ \@for\tname:=#1\do{\cleartheorem\tname} }
