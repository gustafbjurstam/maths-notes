\documentclass[12pt]{article}

\usepackage{../mathsnotes}

\begin{document}
\begin{center}
\textbf{\large Definitions used in Brezis' \textit{Functional Analysis, Sobolev Spaces and Partial Differential Equations} (first edition)} \\
Gustaf Bjurstam\\
bjurstam@kth.se\\
\end{center}
\section*{Preliminaries -- not in the book}
\begin{definition}
    Let $E$ be a vector space over $\real$. A \textit{functional} is a function $f:A\to\real$ where $A$ is some subspace of $E$.
\end{definition}

\section{The Hahn-Banach Theorems. Introduction to Conjugate Convex Functions}
\begin{definition}
    Let $E$ be a vector space over $\real$. A \textit{Minkowski functional} is a function $p:E\to\real$ satisfying
    \begin{align}
        p(\lambda x) &= \lambda p(x), &\forall x\in E \text{ and } \lambda > 0.\\
        p(x+y) &\leq p(x)+p(y), &\forall x,y\in E.
    \end{align}
\end{definition}

\begin{definition}
    Let $P$ be a set with a (partial) order relation $\leq$. A subset $Q\subseteq P$ is \textit{totally ordered} if for any pair $(a,b)$ in $Q$ at least one of $a\leq b$ and $b\leq a$ holds. 
\end{definition}

\begin{definition}
    Let $P$ be a set with a partial order relation $\leq$, and let $Q\subset P$. We say that $c\in P$ is an \textit{upper bound} for $Q$ if $a\leq c$ for all $a\in Q$. We say that $m\in P$ is a \textit{maximal element} of $P$ if there is no element $x\in P\setminus\set{m}$ such that $m\leq x$. If every totally ordered subset $Q$ of $P$ has an upper bound, we call $P$ \textit{inductive}.
\end{definition}



\end{document}
