\documentclass[11pt]{article}

% Basic preamble with commands and packages

% Document related packages
\usepackage[margin=1in, a4paper]{geometry}
\usepackage{parskip}
\usepackage{enumitem}
\usepackage{xcolor}
\baselineskip 1.5em

% AMS packages and other math fonts
\usepackage{amsmath}
\usepackage{amssymb}
\usepackage{amsthm}
\usepackage{mathrsfs}
\usepackage{amsfonts}
\usepackage{bbm}

% Theorem and proof environments
% By default, all theorems, propositions, lemmas, etc. should share a counter
\theoremstyle{plain}
\newtheorem{theorem}{Theorem}[section]
\newtheorem{lemma}[theorem]{Lemma}
\newtheorem{proposition}[theorem]{Proposition}
\newtheorem{corollary}[theorem]{Corollary}

\theoremstyle{definition}
\newtheorem{definition}{Definition}[section]
\newtheorem*{remark}{Remark}

% Shortcuts
% Specific sets
\newcommand{\N}{\mathbb{N}}
\newcommand{\real}{\mathbb{R}}
\newcommand{\rat}{\mathbb{Q}}
\newcommand{\eps}{\varepsilon}
\newcommand{\C}{\mathbb{C}}
\newcommand{\D}{\mathbb{D}}
\renewcommand{\H}{\mathbb{H}}
\newcommand{\Z}{\mathbb{Z}}
\renewcommand{\empty}{\varnothing}

% Various mathscr/mathcal etc, algebras etc.
\newcommand{\F}{\mathscr{F}}    % Fourier
\newcommand{\algebra}{\mathscr{A}}
\newcommand{\Lscr}{\mathscr{L}}
\newcommand{\Lcal}{\mathcal{L}}
\newcommand{\Acal}{\mathcal{A}}
\newcommand{\Ecal}{\mathcal{E}}


% Parentases, sets, norms, etc.
\newcommand{\norm}[1]{\left\lVert#1\right\rVert}
\newcommand{\inner}[2]{\left\langle#1, #2\right\rangle}
\newcommand{\abs}[1]{\left|#1\right|}
\newcommand{\set}[1]{\left\{#1\right\}}
\newcommand{\Fo}[1]{\F\left[#1\right]} % Fourier transform
\newcommand{\Foi}[1]{\F^{-1}\left[#1\right]} % Inverse fourier
\newcommand{\E}[1]{\mathbb{E} \left[#1\right]} % Expected value

% Other notations
\newcommand{\closure}[1]{\overline{#1}}
\newcommand{\dual}{{}^{\star}}
\newcommand{\bidual}{{}^{\star\star}}

% Operators
\DeclareMathOperator{\cond}{cond}
\DeclareMathOperator{\range}{range}
\DeclareMathOperator{\domain}{domain}
\DeclareMathOperator{\argmin}{argmin}
\DeclareMathOperator{\prox}{prox}
\DeclareMathOperator{\TV}{TV}
\DeclareMathOperator{\TGV}{TGV^2_{\alpha,\beta}}
\DeclareMathOperator{\BGV}{BGV^2_{\alpha,\beta}}
\let\div\relax
\DeclareMathOperator{\div}{div}
\DeclareMathOperator{\var}{Var}
\DeclareMathOperator{\cov}{Cov}
\DeclareMathOperator{\rank}{rank}
\DeclareMathOperator{\epi}{epi}

% Make it possible to redefine things
% theorems
\makeatletter
\def\cleartheorem#1{%
  % Undefine the environment start command (e.g., \lemma)
  \expandafter\let\csname#1\endcsname\relax
  % Undefine the environment end command (e.g., \endlemma)
  \expandafter\let\csname end#1\endcsname\relax
  % Undefine the counter (e.g., \c@lemma)
  \expandafter\let\csname c@#1\endcsname\relax
  % Undefine the numbering macro (e.g., \thelemma) - THIS IS THE MISSING PART
  \expandafter\let\csname the#1\endcsname\relax
}
\makeatother
\def\clearthms#1{ \@for\tname:=#1\do{\cleartheorem\tname} }


\begin{document}
\begin{center}
\textbf{\large Definitions used in Brezis' \textit{Functional Analysis, Sobolev Spaces and Partial Differential Equations} (first edition)} \\
Gustaf Bjurstam\\
bjurstam@kth.se\\
\end{center}
\section*{Preliminaries -- not in the book}
\begin{definition}
    Let $E$ be a vector space over $\real$. A \textit{functional} is a function $f:A\to\real$ where $A$ is some subspace of $E$.
\end{definition}

\section{The Hahn-Banach Theorems. Introduction to Conjugate Convex Functions}
\begin{definition}
    Let $E$ be a vector space over $\real$. A \textit{Minkowski functional} is a function $p:E\to\real$ satisfying
    \begin{align}
        p(\lambda x) &= \lambda p(x), &\forall x\in E \text{ and } \lambda > 0.\\
        p(x+y) &\leq p(x)+p(y), &\forall x,y\in E.
    \end{align}
\end{definition}

\begin{definition}
    Let $P$ be a set with a (partial) order relation $\leq$. A subset $Q\subseteq P$ is \textit{totally ordered} if for any pair $(a,b)$ in $Q$ at least one of $a\leq b$ and $b\leq a$ holds. 
\end{definition}

\begin{definition}
    Let $P$ be a set with a partial order relation $\leq$, and let $Q\subset P$. We say that $c\in P$ is an \textit{upper bound} for $Q$ if $a\leq c$ for all $a\in Q$. We say that $m\in P$ is a \textit{maximal element} of $P$ if there is no element $x\in P\setminus\set{m}$ such that $m\leq x$. If every totally ordered subset $Q$ of $P$ has an upper bound, we call $P$ \textit{inductive}.
\end{definition}

\begin{definition}
    Let $E$ be a real normed vector space. We denote by $E\dual$ the \textit{dual space} of $E$, that is, the set of all continuous linear functionals on $E$. The \textit{dual norm} is defined by
    \begin{equation*}
        \norm{f}_{E\dual} = \sup_{\substack{x\in E\\ \norm{x} \leq 1}} f(x).
    \end{equation*}
    Given $f\in E\dual$ and $x\in E$ we may write $\inner{f}{x}$ instead of $f(x)$; we say that $\inner{}{}$ is the \textit{scalar product for the duality} $E\dual, E$. 
\end{definition}

\begin{definition}
    Let $E$ be a normed vector space over $\real$. For every $x_0\in E$, we set 
    \begin{equation*}
        F(x_0) = \set{f_0\in E\dual : \norm{f_0} = \norm{x_0} \text{ and } \inner{f_0}{x_0} = \norm{x_0}^2}.
    \end{equation*}
    The map $x_0\mapsto F(x_0)$ is called the \textit{duality map} of $E$ into $E\dual$.
\end{definition}

\begin{definition}
    Let $E$ be a real vector space. An \textit{affine hyperplane} is a subset $H$ of $E$ of the form $H= \set{x\in E : f(x) = \alpha}$ where $f$ is a linear functional not necessarily in $E\dual$, and $\alpha \in \real$ is a given constant.  We write $H=[f=\alpha]$ and say that $f=\alpha$ is the equation of $H$.
\end{definition}

\begin{definition}
    Let $E$ be a normed vector space. Let $A,B\subset E$, we say that the hyperplane $H=[f=\alpha]$ \textit{separates} $A$ and $B$ if $f(x)\leq \alpha$ for all $x\in A$ and $f(x)\geq \alpha$ for all $x\in B$. If there is $\eps>0$ such that $f(x)\leq \alpha-\eps, \forall x\in A$ and $f(x)\geq \alpha+\eps, \forall x\in B$, we say that $H$ \textit{strictly separates} $A$ and $B$.  
\end{definition}

\begin{definition}
    Let $E$ be a normed vector space. We say that $A\subset E$ is convex if $tx +(1-t)x\in A$ for all $x,y\in A$ and $t\in [0,1]$.
\end{definition}

\begin{definition}
    Let $E$ be a normed vector space, and let $C\subset E$ be an open convex set with $0\in C$. For every $x\in E$ set $p(x)= \inf \set{\alpha : \alpha^{-1}x\in C}$. We call $p$ the \textit{gauge of $C$} or the \textit{Minkowski functional of $C$}.
\end{definition}

\begin{definition}
    Let $E$ be a normed vector space. The \textit{bidual} $E\bidual$ is the dual of $E\dual$ with norm
    \begin{equation*}
        \norm{\xi}_{E\bidual} = \sup_{\substack{f\in E\dual\\ \norm{f} \leq 1}} \inner{\xi}{f}.
    \end{equation*}
    The \textit{canonical injection} $J:E\to E\bidual$ is defined as $Jx = \inner{f}{x}$, this 
\end{definition}

\begin{definition}
    An \textit{isometry} is a map $f$ between metric spaces $A$ and $B$ such that $d_A(x,y) = d_B(f(x),f(y))$ for all $x,y\in A$.
\end{definition}

\begin{definition}
    Let $E$ be a normed vector space. If the canonical injection $J:E\to E\bidual$ is surjective, then we say that $E$ is \textit{reflexive} and identify $E$ with $E\bidual$.
\end{definition}

\begin{definition}
    Let $E$ be a normed vector space, and suppose $M$ is a linear subspace of $E$. We set $$M^\perp = \set{f\in E\dual : \inner{f}{x}=0 \forall x\in M}.$$ If $N$ is a linear subspace of $E\dual$, we set $$N^\perp = \set{x\in E: \inner{f}{x}=0 \forall f\in N}.$$ Note that $N^\perp$ is a subset of $E$ and not $E\bidual$. We call $M^\perp$ (or $N^\perp$) \textit{the space orthogonal to $M$ (or $N$)}.
\end{definition}

\begin{definition}
    Let $E$ be a set, and let $\varphi:E\to (-\infty,\infty]$. We denote $D(\varphi) = \set{x\in E: \varphi(x)<\infty}$, and define the \textit{epigraph} of $\varphi$ as $\epi\varphi = \set{[x, \lambda]\in E \times \real:\varphi(x)\leq \lambda }$.
\end{definition}
\end{document}
