\documentclass[12pt]{article}

% Load AMS packages for advanced math
\usepackage{amsmath,amssymb,amsthm,mathrsfs}
\usepackage{pgfplots}
\usepackage{enumitem}
\pgfplotsset{compat=newest}

% Load geometry package and set margins
\usepackage[margin=1in]{geometry}

\newtheorem{lemma}{Lemma}
\newtheorem{sats}{Theorem}

\theoremstyle{definition}
\newtheorem{definition}{Definition}
\newtheorem*{remark}{Remark}

% Shortcuts
\newcommand{\D}{\mathbb{D}}      % unit disc
\renewcommand{\H}{\mathbb{H}}      % Half plane
\newcommand{\N}{\mathbb{N}}      % natural numbers
\newcommand{\Z}{\mathbb{Z}}
\newcommand{\real}{\mathbb{R}}   % real numbers
\newcommand{\rat}{\mathbb{Q}}     % rationals
\newcommand{\eps}{\varepsilon}    % for nice epsilon
\newcommand{\sol}{ \noindent\textbf{Solution: } }   % creates Solution:
\newcommand{\prob}[1]{ \noindent\textbf{Problem #1.} }
\newcommand{\C}{\mathbb{C}}    % complex numbers
\renewcommand\part[1]{\vspace{.10in}\textbf{(#1)}}
\newcommand{\algebra}{\mathscr{A}}
\renewcommand{\L}{\mathscr{L}}
\renewcommand{\Re}{\text{Re }}
\renewcommand{\Im}{\text{Im }}
%\newtheorem*{solution}{Solution}
\newenvironment{solution}{\renewcommand{\proofname}{Lösning}\begin{proof}}{\end{proof}}
\newenvironment{bevis}{\renewcommand{\proofname}{Bevis}\begin{proof}}{\end{proof}}
\newcommand{\exempel}[1]{ \noindent\textbf{Exempel #1.} }
\usepackage{parskip}

\newcommand\norm[1]{\left\lVert#1\right\rVert}

%%%%%%%%%%%%%%%%%%%%%%%%%%%%%%%%%%%%%%%%%%%
\begin{document}

% Set linespacing to 1.5
\baselineskip 1.5em

\begin{center}
\textbf{\large Statements of Examinable Theorems \\ SF2745 Advanced Complex Analysis} \\
Gustaf Bjurstam\\
bjurstam@kth.se\\
\end{center}

\begin{remark}
    Theorem's for which the proof must be supplied are marked with $(*)$.
\end{remark}

\begin{sats}
    Under stereographic projection, circles and lines in $\C$ correspond to circles in $\mathbb{S}^2$.
\end{sats}

\begin{sats}[M test {$*$}]
    If $|a_n(z-z_0)^n|\leq M_n$ for $|z-z_0|\leq r$ and if $\sum M_n<\infty$, then $\sum a_n(z-z_0)^n$ converges absolutely and uniformly in $\{z:|z-z_0|\leq r\}$.
\end{sats}

\begin{sats}[Root test]
    Suppose $\sum a_n(z-z_0)^n$ is a formal power series. Let
    \begin{equation*}
        R=\liminf_{n\to\infty} |a_n|^{-1/n}=\frac{1}{\limsup_{n\to\infty} |a_n|^{1/n}}\in [0, \infty].
    \end{equation*}
    Then $\sum a_n(z-z_0)^n$
    \begin{enumerate}[label=(\alph*)]
        \item coverges absolutely in $D(R,z_0)$,
        \item converges uniformly in $\overline{D(r,z_0}$ for all $r\in(0,R)$,
        \item diverges in $\C\setminus\overline{D(R,z_0)}$.
    \end{enumerate}
\end{sats}

\begin{sats}[Uniqueness $*$]
    If $f$ is analytic in a region $\Omega$ then either $f\equiv 0$ or the zeros of $f$ are isolated.
\end{sats}

\begin{sats}[Power series expansion $*$]
An analytic function $f$ has derivatives of all orders. If $f$ is equal to a convergent power series on $D(r,z_0)$ then the power series is given by 
\begin{equation*}
    f(z)=\sum_{n\geq 0} \frac{f^{(n)}(z_0)}{n!}(z-z_0)^n,
\end{equation*}
for $z\in D(r,z_0)$.
\end{sats}

\begin{sats}[Maximum principle]
    If $f$ is analytic on a region $\Omega$, then unless $f$ is constant, there is no $z_0\in\Omega$ such that $f(z_0)=\sup_{z\in\Omega} f(z)$.
\end{sats}

\begin{sats}[Openness of NC analytic maps]
    If $E\subseteq \Omega$ is an open set, and $f$ is a non-constant analytic map on $\Omega$, then $f(E)$ is open. 
\end{sats}

\begin{sats}[Liouville $*$]
    If $f$ is an entire function, and $f$ is bounded, then $f$ is constant.
\end{sats}

\begin{sats}[Schwarz lemma $*$]
    If $f$ is analytic on $\D$, $f(0)=0$, and $|f(z)|\leq 1$ on $\D$, then $|f(z)|<|z|$ for all $z\in\D$, and $f'(0)\leq 1$. Moreover, if $|f(z)|=|z|$ for any non-zero $z\in \D$, or if $|f'(0)|=1$, then there is $c\in\partial \D$ such that $f(z)=cz$.  
\end{sats}

\begin{sats}[$*$]
    Let $C$ be a positivley oriented circle of radius $r$ centered at $z_0$, then
    \begin{equation*}
        \int_C \frac{dz}{z-a}=\begin{cases}
            1,\quad |a-z_0|<r,\\
            0,\quad |a-z_0|>r.
        \end{cases}
    \end{equation*}
\end{sats}

\begin{sats}[Cauchy simple $*$]
    Let $D$ be a disc of radius $r$ centered at $z_0$, and suppose $f$ is holomorphic on $D$, then $$f(z)=\frac{1}{2\pi i}\int_{\partial D} \frac{f(w)}{w-z}dw$$ for all $z\in D$. 
\end{sats}

\begin{sats}
    A function $f$ is holomorphic on a region $\Omega$ if and only if it is analytic on $\Omega$. Moreover, the series expansion for $f$ centered at $z_0\in \Omega$ has the radius of the largest disc centered at $z_0$ and contained in $\Omega$ as its radius of convergence.
\end{sats}

\begin{sats}[Cauchy estimate and derivatives $*$]
    Let $D$ be a disc of radius $r$ centered at $z_0$, and suppose $f$ is holomorphic on $D$, then $$f^{(n)}(z_0)=\frac{n!}{2\pi i}\int_{\partial D} \frac{f(w)}{(w-z_0)^{n+1}}dw$$ for all $n\geq 0$. And $$|f^{(n)}(z_0)|\leq\frac{n!}{r^n}\sup_{w\in\partial D} |f(w)|$$.
\end{sats}

\begin{sats}[Morea]
    If $f$ is continuous in an open disc $B$, and if $$\int_{\partial R} f \, dz=0$$ for all closed rectangels $R\subset B$ with sides parallel to the axes, then $f$ is analytic on $B$. 
\end{sats}

\begin{sats}[Runge]
    If $f$ is analytic on a compact set $K$, and if $\eps>0$, then there is a rational function $r$ so that 
    \begin{equation*}
        \sup_{z\in K}|f(z)-r(z)|<\eps.
    \end{equation*}
\end{sats}

\begin{sats}[Weirstrass]
    Suppose $\{f_n\}$ is a collection of analytic functions on a regiona $\Omega$ such that $f_n\to f$ uniformly on compact subsets of $\Omega$. Then $f$ is analytic in $\Omega$. Moreover $f_n'\to f'$ uniformly on compact subsets of $\Omega$.
\end{sats}

\begin{sats}[Cauchy integral formula]
    Suppose $\gamma$ is cycle contained in a region $\Omega$, and suppose $$\int_\gamma \frac{dw}{w-a}=0$$ for all $a\notin\Omega$. If $f$ is analytic on $\Omega$ and $z\in \C\setminus\gamma$ then $$\int_\gamma \frac{f(w)}{w-z}\,dw=f(z)\int_\gamma \frac{dw}{w-z}.$$
\end{sats}

\begin{sats}[Primitive, log, on SCR]
    Suppose $f$ is analytic on a simply-connected region $\Omega$. Then 
    \begin{enumerate}[label=(\roman*)]
        \item $\int_\gamma f = 0$ for all closed curves $\gamma\subset\Omega$,
        \item there exists a function $F$ analytic on $\Omega$ such that $F'=f$,
        \item if also $f\neq 0$ for all $z\in\Omega$, then there exists a function $g$ on $\Omega$ such that $f=e^g$.
    \end{enumerate}
\end{sats}

\begin{sats}[Laurent]
    Suppose $f$ is analytic on $A=\{z:r<|z-a|<R\}$. Then there is a unique sequence $\{a_n\}\subset\C$ so that
    \begin{equation*}
        f(z)=\sum_{n\in\Z} a_n(z-a)^n,
    \end{equation*}
    where the series converges uniformly and absolutely on compact subsets of $A$. Moreover, 
    \begin{equation*}
        a_n=\frac{1}{2\pi i}\int_{C_s} \frac{f(w)}{(w-a)^{n+1}}\,dw,
    \end{equation*}
    where $C_s$ is a positively oriented circle centered at $a$ with radius $s\in(r,R)$.
\end{sats}

\begin{sats}[Density near essential $*$]
    If $f$ is analytic in $U=\{z:0<|z-b|<\delta\}$ and if $b$ is an essential singularity of $f$, them $f(U)$ is dense in $\C$.
\end{sats}

\begin{sats}[Argument Principle $*$]
    Suppose $f$ is meromorphic in a region $\Omega$, with zeros $z_j$ and poles $p_k$. Suppose $\gamma$ is a cycle with $\gamma\sim 0$ in $\Omega$, and suppose no zero or pole appears on $\gamma$. Then
    \begin{equation*}
        n(f(\gamma),0)=\frac{1}{2\pi i}\int_\gamma\frac{f'}{f} = \sum_j n(\gamma,z_j)-\sum_k n(\gamma,p_k).
    \end{equation*}
\end{sats}

\begin{sats}[LFT circles]
    LFTs map "circles" onto "circles" and "discs" onto "discs".
\end{sats}

\begin{sats}[LFT three points $*$]
    Given $z_1,z_2,z_3$, distinct points in $\C^*$ and $w_1,w_2,w_3$ distinct points in $\C^*$, there is a unique $LFT$, T, such that $T(z_i)=w_i$, for $i=1,2,3$.
\end{sats}

\begin{sats}[Max principle $*$]
    Let $u$ be subharmonic on a region $\Omega$. Then, unless $u$ is constant, there is no $z\in\Omega$ such that $u(z)=\sup_{w\in\Omega}u(w)$.
\end{sats}

\begin{sats}[Poisson integral]
    Let $g$ be a continuous function defined on $\partial\D$, the solution to the Dirichlet problem on $\D$ with boundary values $g$ is given by
    \begin{equation*}
        u(z)=PI(g)(z)=\int_0^{2\pi} P_z(t)g(e^{it})\,dt=\int_{0^{2\pi}}\frac{1}{2\pi}\frac{1-|z|^2}{|e^{it}-z|^2}g(e^{it})\,dt.
    \end{equation*}
\end{sats}

\begin{sats}[Cauchy-Riemann]
    Suppose $\Omega$ is a region, and that $u,v:\Omega\to\real$ are $C^1$. Then $u$ and $v$ satisfy the Cauchy-Riemann equations 
    \begin{equation*}
        u_x=v_y,\qquad u_y=-v_x,
    \end{equation*}
    if and only if $f=u+iv$ is holomorphic on $\Omega$.
\end{sats}

\begin{sats}[Lindelöf max $*$]
    Suppose $\Omega$ is a region and that $\{\xi_1,\dots,\xi_n\}$ is a finite subset of $\partial\Omega$, not equal to $\partial\Omega$. If $u$ is subharmonic on $\Omega$ with $u\leq M<\infty$ on $\Omega$ and if 
    \begin{equation*}
        \limsup_{z\in\Omega\to\xi} u(z) \leq m,
    \end{equation*}
    for all $\xi\in\Omega\setminus\{\xi_1,\dots,\xi_n\}$, then $u\leq m$ on $\Omega$.
\end{sats}

\begin{sats}[Harnack inequality]
    Suppose $u$ is a positive harmonic function on $\D$. Then
    \begin{equation*}
        \frac{1-r}{1+r}u(0)\leq u(z)\leq \frac{1+r}{1-r}u(0),
    \end{equation*}   
    for $r=|z|<1$.
\end{sats}

\begin{sats}[Harnack's principle]
    Suppose $\{u_n\}$ are harmonic on a region $\Omega$ such that $u_n(z)\leq u_{n+1}(z)$ for all $z\in\Omega$. Then either
    \begin{enumerate}[label=(\roman*)]
        \item $\lim_n u_n = u$ exists and is harmonic on $\Omega$, or
        \item $\lim_n u_n = \infty$ everywhere,
    \end{enumerate}
    where convergence is uniform on compact subsets of $\Omega$. In case (ii), this means that given compact $K\subset \Omega$ and $M<\infty$, there is $n_0$ such that $u_n>m$ for all $z\in K$, and $n>n_0$.
\end{sats}

\begin{sats}[Schwarz reflection]
    Suppose $\Omega$ is a region which is symmetric about $\real$. Set $\Omega^+=\Omega\cap\H$ and $\Omega^-=\Omega\setminus\overline{\H}$. If $\nu$ is harmonic on $\Omega^+$, continuous on $\Omega^+\cup(\Omega\cap\real)$ and equal to $0$ on $\Omega\cap\real$, then the function defined by
    \begin{equation*}
        V(z)=\begin{cases}
            \nu(z),&z\in\Omega\setminus\Omega^-\\
            -\nu(\overline{z}),&z\in \Omega^-
        \end{cases}
    \end{equation*}
    is harmonic on $\Omega$. If also $\nu(z)=\Im f(z)$, where $f$ is analytic on $\Omega^+$, then the function
    \begin{equation*}
        g(z)=\begin{cases}
            f(z),&z\in\Omega^+\\
            \overline{f(\overline{z})},&z\in \Omega^-
        \end{cases}
    \end{equation*}
    extends to be analytic in $\Omega$.
\end{sats}

\begin{sats}[Schwarz-Christoffel]
    Suppose $\Omega$ is a bounded simply-connected region whose positively oriented boundary $\partial\Omega$ is a polygon with verticies $v_1,\dots,v_n$. Suppose the tangen direction on $\partial\Omega$ increases by $\pi\alpha_j$ at vertex $v_j$, $\alpha_j\in(-1,1)$. Then there exists $x_1<x_2<\dots<x_n$ and constants $c_1,c_2$ so that
    \begin{equation*}
        f(z) = c_1\int_{\gamma_z} \prod_{j=1}^n (\zeta-x_j)^{-\alpha_j}\,d\zeta +c_2
    \end{equation*}
    is a conformal map of $\H$ onto $\Omega$, where the integral is along any curve $\gamma_z$ in $\H$ from $i$ to $z$.
\end{sats}

\begin{sats}[Residue Theorem $*$]
    Suppose $f$ is analytic in $\Omega$ except for isolated singularities at $a_1,\dots,a_n$. If $\gamma$ is a cycle in $\Omega$ with $\gamma\sim 0$ and $a_j\notin\gamma$, $j=1,\dots,n$, then
    \begin{equation*}
        \int_\gamma f=2\pi i\sum_{k} n(\gamma,a_k)\text{Res}_{a_k} f.
    \end{equation*}
\end{sats}

\begin{sats}[Arzela-Ascoli]
    A family $\mathcal{F}$ of continuous functions is normal on a region $\Omega\subset\C$ if and only if
    \begin{enumerate}[label=(\roman*)]
        \item $\mathcal{F}$ is equicontinuous on $\Omega$, and
        \item there is a $z_0\in \Omega$ so that the collection $\{f(z_0):f\in\mathcal{F}\}$ is a bounded subset of $\C$.
    \end{enumerate}
\end{sats}

\begin{sats}[Normal analytic families]
    The following are equivalent for a family $\mathcal{F}$ of analytic functions on a region $\Omega$
    \begin{enumerate}[label=(\roman*)]
        \item $\mathcal{F}$ is normal on $\Omega$;
        \item $\mathcal{F}$ is locally bounded on $\Omega$;
        \item $\mathcal{F}'=\{f':f\in\mathcal{F}\}$ is locally bounded on $\Omega$ and there is a $z_0\in\Omega$ so that $\{f(z_0):f\in\mathcal{F}\}$ is a bounded subset of $\C$. 
    \end{enumerate}
\end{sats}

\begin{sats}[Riemann Mapping Theorem $*$]
    Suppose $\Omega\subset\C$ is simply-connected and $\Omega\neq\C$. Then there exists a one-to-one map $f$ of $\Omega$ onto $\D$. If $z_0\in\Omega$, then there is a unique such map with $f(z_0)=0$ and $f'(z_0)>0$.
\end{sats}

\begin{sats}[Mittag-Leffler]
    Suppose $b_k\in\Omega\to\partial\Omega$ with $b_k\neq b_j$ if $k\neq j$. Set 
    \begin{equation*}
        S_k(z)=\sum_{j=1}^{n_k}\frac{c_{j,k}}{(z-b_k)^j},
    \end{equation*}
    where each $n_k$ is a positive integer and $c_{j,k}\in\C$. Then there is a function meromorphic in $\Omega$ with singular parts $S_k$ at $b_k$, $k=1,2,\dots$, and no other singularities in $\Omega$. 
\end{sats}

\begin{sats}[Weierstrass Product Theorem]
    Suppose $\Omega$ is a bounded region. If $\{b_j\}\subset\Omega$ with $b_j\to\partial\Omega$, and if $n_j$ are positive integers, then there exists an analytic function $f$ on $\Omega$ such that $f$ has a zero of order exactly $n_j$ at $b_j$, $j=1,2,\dots$, and no other zeros in $\Omega$.
\end{sats}

\begin{sats}[Jensen's formula]
    Suppose $f$ is meromorphic on $|z|\leq R$ with zeros $\{a_k\}$ and poles $\{b_j\}$. Suppose also that $0$ is not a zero or a pole of $f$. Then
    \begin{equation*}
        \frac{1}{2\pi}\int_{-\pi}^\pi \log|f(Re^{it})|\,dt=\log|f(0)|+\sum_{a_k<R}\log\frac{R}{|a_k|}-\sum_{b_j<R}\log\frac{R}{|b_j|}.
    \end{equation*}
\end{sats}

\begin{sats}[Monodromy]
    Suppose $\Omega$ is simply-connected and suppose $f_0$ is defined and anakytic in a neighbourhood of $b\in\Omega$. If $f_0$ can be analytically continued along all curves in $\Omega$ beginning at $b$ then there is an analytic function $f$ on $\Omega$ so that $f=f_0$ in a neighbourhood of $b$.
\end{sats}

\begin{sats}[Green's function $*$]
    Suppose $p_0\in W$ and suppose $z:U\to\D$ is a coordinate function such that $z(p_0)=0$. If $g_W(p,p_0)$ is exists, then 
    \begin{align*}
        &g_W(p,p_0)>0\quad \text{for } p\in W\setminus\{p_0\},\text{ and,}\\
        &g_W(p,p_0)+\log |z(p)|\quad\text{extends to be harmonic in } U.
    \end{align*}
\end{sats}

\begin{sats}[Green is symmetric if $W=\D$]
    Suppose $W$ is a Riemann surface for which Green's function $g_W$ with pole at $p$ exists, for some $p\in W$, and suppose $W^*=\D$. Then $g_W$ with pole at $q$ exists for all $q\in W$, and
    \begin{equation*}
        g_W(p,q)=g_W(q,p).
    \end{equation*}
\end{sats}

\begin{sats}[Uniformisation case 1 $*$]
    If $W$ is a simply-connected Riemann surface then the following are equivalent:
    \begin{enumerate}[label=(\roman*)]
        \item $g_W(p,p_0)$ exists for some $p_0\in W$,
        \item $g_W(p,p_0)$ exists for all $p_0\in W$, and
        \item there is a one-to-one analytic map $\varphi$ of $W$ onto $\D$.
    \end{enumerate}
    Moreover, if $g_W$ exists, then
    \begin{equation*}
        g_W(p_1,p_0)=g_W(p_0,p_1),
    \end{equation*}
    and $g_W(p,p_0)=-\log|\varphi(p)|$, where $\varphi(p_0)=0$.
\end{sats}

\begin{sats}[Green is symmetric general case]
    Suppose $W$ is a Riemann surface for which Green's function $g_W$ with pole at $p$ exists, for some $p\in W$. Then $g_W$ with pole $q$ exists, for all $q\in W$, and 
    \begin{equation*}
        g_W(p,q)=g_W(q,p).
    \end{equation*}
\end{sats}

\begin{sats}[Uniformisation case 2]
    Suppose $W$ is a simply-connected Riemann surface for which Green's function does not exist. If $W$ is compact, then there is a one-to-one analytic map of $W$ onto $\C^*$. If $W$ is not compact, then there is a one-to.one analytic moap of $W$ onto $\C$.
\end{sats}

\end{document}
